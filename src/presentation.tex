\section{Présentation de la solution proposée}
\label{sec:presentation}

Comme annoncé dans l'introduction, nous avons cherché une solution
permettant de résoudre au mieux le problème de recherche de place.

L'idée de départ est que nous souhaitions qu'un agent seul puisse
trouver une place dans des temps raisonnables, et cela impliquait que
chaque agent ait un certain rayon de vision de manière à être
autonome.

Nous avons aussi défini un protocole de communication simple pour
prévenir les agents en recherche de place lorsqu'un agent libère une
place. On s'est rendu compte qu'il était plus efficace de prévenir au
moins un tour avant de libérer la place. En pratique, ce protocole ne
prévient qu'un seul agent qui dispose donc d'une information lui
donnant un avantage sur les autres. Il abandonne donc la recherche
d'autres places, et se dirige vers celle qui va se libérer. Ce qui
réduit le temps où la place est libre.

Dans la partie \ref{sec:benchmarks}, nous présenterons quelques
résultats lorsque nous n'utilisons que la vision et nous les
comparerons avec les résultats avec les messages.

%%% Local Variables:
%%% mode: latex
%%% TeX-master: "../sma"
%%% compile-command: "cd ..; make"
%%% End:
