\section{Présentation de la solution proposée}
\label{sec:presentation}

\subsection{Concept}
\label{sec:concept}

Comme annoncé dans l'introduction, nous avons cherché une solution
permettant de résoudre au mieux le problème de recherche de place.

Rappelons les règles du projet, il y a un nombre $n$ de places sur une
grille de taille indéfinie, où chaque case peut éventuellement être
une case. Il y a un nombre $p$ d'agents, avec $n < p$. Lorsqu'un agent
a trouvé une place, il y reste un temps fixé, puis lorsqu'il sort,
passe un autre temps fixé à se déplacer aléatoirement sans rechercher
de places.

Notre idée de départ est que nous souhaitions qu'un agent seul puisse
trouver une place dans des temps raisonnables, et cela impliquait que
chaque agent ait un certain rayon de vision de manière à être
autonome.

Nous avons aussi défini un protocole de communication simple pour
prévenir les agents en recherche de place lorsqu'un agent libère une
place. On s'est rendu compte qu'il était plus efficace de prévenir au
moins un tour avant de libérer la place. En pratique, ce protocole ne
prévient qu'un seul agent qui dispose donc d'une information lui
donnant un avantage sur les autres. Il abandonne donc la recherche
d'autres places, et se dirige vers celle qui va se libérer. Ce qui
réduit le temps où la place est libre.

% Dans la partie \ref{sec:benchmarks}, nous présenterons quelques
% résultats lorsque nous n'utilisons que la vision et nous les
% comparerons avec les résultats avec les messages en plus.

Nous avons souhaité conserver une modélisation simple, mélangeant les
avantages des agents réactifs (qui se dirigent vers une place libre
lorsqu'il y en a une), et des agents cognitifs (qui communiquent pour
améliorer la résolution du problème).


\subsection{Implémentation}
\label{sec:implementation}

Pour mettre en place notre solution, nous avons développé trois types
d'agents différents:

\begin{description}
\item[RandomAgent] Ils se déplacent de manière aléatoire, changeant de
  direction toutes les 6 itérations. Ils se garent lorsqu'ils sont
  positionnés sur une place. Ils serviront pour comparer notre
  solution lors des benchmarks.
\item[SmartAgent] Ils suivent la description donnée dans la partie
  \ref{sec:concept}.
\item[ParkAgent] Ce sont des agents qui représentent des places. Ils
  n'agissent pas, ni ne prennent pas de réservations. Ils ont
  simplement un membre booléen signalant leur disponibilité ou non.
\end{description}

Il n'y a pas de raison d'entrer plus en détail dans l'implémentation
des RandomAgents et des ParkAgents, dans cette section nous
présenterons plus en détail comment nous avons implémenté SmartAgent,
et quels sont les paramètres qui influent sur le système. Dans la
partie \ref{sec:benchmarks}, nous ferons varier ces paramètres, et
nous analyserons les effets qu'ils ont sur le système.

\subsubsection{Le code}
\label{sec:code}

Le code peut être divisé en deux parties distinctes, les deux méthodes
représentant les deux principaux états de l'agent, et une série de méthodes
pour simplifier le travail de celles-ci.

\paragraph{Live}

La méthode ``live'' prend en charge la période de recherche et celle
du déplacement aléatoire avant de se remettre à chercher.

Lorsque l'on est dans l'état aléatoire, on ne fait rien à part changer
de direction si on l'a fait il y a suffisamment longtemps, et avancer
d'une case.

Sinon, on vide la boite au lettre, et s'il y a un message on le lit et
on affecte notre cible avec la valeur du message. Si on n'a pas encore
de cible on en cherche une et on se dirige vers elle si on en trouve
une. Sinon on se déplace aléatoirement. Si on a déjà une, on se dirige
vers elle. On vérifie aussi si on est garé, si c'est le cas on change
d'état pour ``parked''.

\paragraph{Parked}

On laisse passer le nombre d'itérations nécessaire pour rester garé.
Les $k$ dernières itérations, on cherche l'agent le plus proche, et
on lui envoie un message pour lui faire savoir que la place va bientôt
se libérer, pour qu'il puisse s'en approcher en vue de s'y garer. La
place n'est pas réservée, puisqu'il est possible que quelqu'un d'autre
arrive et prenne la place avant.

\subsubsection{Les paramètres}
\label{sec:params}

Il y a une liste de paramètres qui influent sur le système
multi-agent, nous allons les présenter ici:

\begin{description}
\item[Parktime] Le temps où chaque agent reste lorsqu'il est garé.
\item[VisionRadius] Le champs de vision de chaque agent.
\item[LapBeforeBroadcast] L'agent garé envoie des messages les $k$
  dernières itérations lorsqu'il est garé. $k$ est ce paramètre.
\item[TimeBeforeSearch] Le temps où l'agent est en aléatoire après
  s'être garé.
\item[NumAgent] Le nombre d'agent sur la carte.
\item[NumPark] Le nombre de place sur la carte.
\end{description}


%%% Local Variables:
%%% mode: latex
%%% TeX-master: "../sma"
%%% compile-command: "cd ..; make"
%%% End:
